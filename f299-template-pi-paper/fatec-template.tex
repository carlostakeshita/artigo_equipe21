%%%% fatec-article.tex, 2024/03/10

%% Classe de documento
\documentclass[
  a4paper,%% Tamanho de papel: a4paper, letterpaper (^), etc.
  12pt,%% Tamanho de fonte: 10pt (^), 11pt, 12pt, etc.
  english,%% Idioma secundário (penúltimo) (>)
  brazilian,%% Idioma primário (último) (>)
]{article}

%% Pacotes utilizados
\usepackage[]{fatec-article}
\usepackage{float}
\Author{1}{Name={Carlos Eduardo Campos Takeshita\\ Yago Uran Kurashiki Rios \\ Marcelo Augusto Pedroso Martins\\ Sthevens Konesuk Miranda dos Santos}}

\Author{2}{Name={\{ carloa.takeshita@fatec.sp.gov.br \}\\ \{ yago.rios@fatec.sp.gov.br \} \\ \{ marcelo.martins36@fatec.sp.gov.br\} \\ \{ sthevens.santos@fatec.sp.gov.br \}}}

%% Definição das palavras-chaves/keywords
\Keyword{1}{Aprendizagem Profunda}{Deep Learning}
\Keyword{2}{Xanthomonas phaseoli}{Xanthomonas phaseoli}
\Keyword{3}{Infecções}{Infections}

%%%% Resumo no idioma primário (brazilian)
\begin{Abstract}[brazilian]%% Idioma (brazilian ou english)
O Projeto Classificação de Bacteriose nas Folhas da Mandioca, orientado por Aprendizagem Profunda, tem como objetivo desenvolver um aplicativo de inteligência artificial (IA) para detectar as probabilidades da infecção por Xanthomonas phaseoli em folhas de mandioca (Manihot esculenta). Utilizando técnicas de Aprendizagem Profunda e HSV para facilitar a identificação visual, o aplicativo permitirá que os usuários capturem imagens das plantas ou enviem imagens armazenadas em seus dispositivos, com intuito de serem analisadas de forma eficaz pela IA do aplicativo, onde será exibido o grau de infecção da planta. Tendo um sistema de mapeamento que ajudará na identificação de áreas mais vulneráveis, possibilitando intervenções direcionadas. A capacidade de detectar infecções de maneira rápida e eficiente é crucial para a diminuição dos danos causados pela bactéria, sendo possível monitorar as plantações através do mobile ou web, utilizando o sistema de calor e as análises das folhas para mapear e prever a incidência da bactéria.
\end{Abstract}

%%%% Resumo no idioma secundário (english)
\begin{Abstract}[english]%% Idioma (brazilian ou english)
The Deep Learning-Driven Bacteriosis Classification in Cassava Leaves Project aims to develop an artificial intelligence (AI) application to detect the probabilities of Xanthomonas phaseoli infection in cassava leaves(Manihot esculenta). Using deep learning techniques and HSV techniques to facilitate visual identification , the application will allow users to capture images of plants or send images stored on their devices, which will be effectively analyzed by the application's AI, where the degree of infection of the plant will be displayed. Furthermore, the project seeks to create a mapping system that will help identify more vulnerable areas, enabling targeted interventions and improving agricultural practices. The ability to detect infections quickly and efficiently is crucial to reducing the damage caused by the bacteria, promoting the sustainability of cassava production and contributing to food security in affected communities.
\end{Abstract}

%% Processamento de entradas (itens) do índice remissivo (makeindex)
\makeindex%

%% Arquivo(s) de referências
\addbibresource{fatec-article.bib}

%% Início do documento
\begin{document}

% Seções e subseções
%\section{Título de Seção Primária}%

%\subsection{Título de Seção Secundária}%

%\subsubsection{Título de Seção Terciária}%

%\paragraph{Título de seção quaternária}%

%\subparagraph{Título de seção quinária}%

\section*{Introdução}%
\label{sect:intro}
A Organização das Nações Unidas (ONU) tem como proposta os Objetivos de Desenvolvimento Sustentável (ODS), que visam promover um equilíbrio entre crescimento econômico, inclusão social e proteção ambiental. Dentre esses objetivos, destaca-se o ODS 15 — “Vida Sobre a Terra”, cujo propósito é proteger, recuperar e promover o uso sustentável dos ecossistemas terrestres, além de gerir de forma sustentável as florestas, combater a desertificação, deter e reverter a degradação da terra e conter a perda da biodiversidade. Essa meta reforça a importância da preservação dos recursos naturais e a necessidade de práticas agrícolas mais sustentáveis.

O Brasil possui grande destaque como um dos principais países exportadores agrícolas do planeta, representando um papel essencial na segurança alimentar global. Em 2023, o país registrou uma alta de 4,8\% nas exportações agrícolas em relação ao ano anterior, atingindo um valor total de US\$166,55 bilhões, segundo dados do Ministério da Agricultura e Pecuária \cite{mudancas2024}. Essa expansão demonstra a relevância e o dinamismo do agronegócio brasileiro, que se apoia tanto em tecnologias inovadoras quanto em práticas tradicionais de cultivo.
Pesquisas realizadas pela Companhia Nacional de Abastecimento (CONAB) — instituição que tem como finalidade regular o estoque de alimentos, auxiliar a agricultura familiar e controlar oscilações de renda — mostram que o Brasil é o quarto maior exportador de mandioca na América Latina. O Estado de São Paulo, por sua vez, apresenta uma área plantada de 51 mil hectares, posicionando-se como o sexto maior produtor de mandioca do país.

Na região do Vale do Ribeira, localizada no sul do estado, encontra-se um número significativo de comunidades produtoras de mandioca, especialmente entre povos quilombolas, que cultivam e comercializam a produção de forma local. Essas comunidades estão distribuídas ao longo da Bacia do Rio Ribeira de Iguape, em áreas remotas e cobertas pela vegetação da Mata Atlântica, considerada uma das regiões de grande biodiversidade do planeta. Desde os primórdios da ocupação, ainda no século XVIII, os quilombolas têm mantido uma relação histórica de dependência e sustentabilidade com o cultivo de alimentos essenciais, como arroz, milho, mandioca e feijão, garantindo sua subsistência e preservando práticas agrícolas tradicionais que reforçam o equilíbrio entre o homem e o meio ambiente.
As bacterioses — doenças originadas por bactérias fitopatogênicas — ocupam posição de destaque entre as pragas agrícolas, sendo capazes de causar grandes prejuízos às lavouras quando não são rapidamente identificadas e controladas \cite{fialho1999}. Entre os diversos patógenos, destaca-se o \textit{Xanthomonas phaseoli pv. manihotis}, que atrai atenção significativa da comunidade científica em virtude de seu comportamento vascular e sistêmico, características que lhe conferem um alto potencial de causar danos severos às plantações de mandioca, especialmente em regiões tropicais. Essa bactéria tem despertado relevância científica mundial devido ao impacto direto que exerce sobre a produção alimentar e econômica \cite{mansfield2012}.

Estudos indicam que a bactéria \textit{Xanthomonas} afeta aproximadamente 30\% da produção global de mandioca, podendo reduzir em até 70\% o peso das raízes quando o controle da doença não é realizado adequadamente \cite{fialho1999}. Tais perdas representam um grave obstáculo à sustentabilidade agrícola e à segurança alimentar em diversas regiões produtoras. Um dos principais desafios enfrentados atualmente é a escassez de informações detalhadas sobre a incidência dessas bactérias, uma vez que, embora existam outros patógenos capazes de infectar a mandioca, a literatura científica ainda é limitada quanto à extensão e ao comportamento dessas infecções \cite{chaves2021}.
Com base nas pesquisas apresentadas, torna-se evidente a preocupação crescente com a ocorrência de bactérias em plantações agrícolas e seus impactos diretos na produção e na economia do setor. Com o objetivo de identificar a presença da bactéria e realizar o mapeamento de sua incidência, foi proposta a criação de um aplicativo apoiado em Redes Neurais Artificiais (RNA).

A Aprendizagem Profunda, subárea da IA, consiste em técnicas capazes de treinar redes neurais profundas para reconhecer padrões complexos e aprender continuamente a partir de grandes volumes de dados. Após o processo de treinamento, essas redes se tornam aptas a executar tarefas semelhantes às realizadas por seres humanos, como a análise e interpretação de imagens com alta precisão.
Assim, o aplicativo desenvolvido tem como finalidade monitorar o estado das plantas de mandioca e identificar, com base em probabilidade, o grau de infecção pela bactéria \textit{Xanthomonas}, fornecendo informações que auxiliem no controle e na prevenção da doença. Essa abordagem tecnológica representa um avanço significativo no monitoramento agrícola, contribuindo para minimizar perdas de produção, aumentar a eficiência no manejo de pragas e promover práticas agrícolas mais sustentáveis, em consonância com os Objetivos de Desenvolvimento Sustentável da ONU.

\newpage

\section*{OBJETIVO} \label{sect:obj}

O presente estudo tem como objetivo principal o desenvolvimento de um sistema baseado em Redes Neurais para identificar e mapear a infecção por \textit{Xanthomonas phaseoli pv. manihotis} em folhas de mandioca. Com base nesse objetivo geral, foram definidos os objetivos específicos do projeto, que buscam detalhar as etapas e resultados esperados da pesquisa:

\begin{enumerate}
\item Desenvolver um sistema baseado em Redes Neurais capaz de realizar a detecção e classificação do grau de infecção das folhas de mandioca, utilizando técnicas de Aprendizagem Profunda para análise de imagens com alta precisão.
\item Integrar os dados de infecção com informações geográficas, possibilitando o mapeamento da distribuição da bactéria na região do Vale do Ribeira e fornecendo subsídios para o monitoramento territorial da praga.
\item Desenvolver um aplicativo interativo, que permita aos usuários capturar ou enviar imagens das folhas e visualizar, de forma intuitiva, o grau de infecção detectado pelo sistema.
\item Fornecer informações estratégicas para o monitoramento e controle da bactéria em tempo real, contribuindo para a redução de danos nas plantações, a manutenção da produtividade agrícola e o suporte à tomada de decisão de agricultores e técnicos da região.
\end{enumerate}

\newpage

\section*{ESTADO DA ARTE} \label{sect:estadoarte}

O uso da IA para resolução de problemas complexos vem ganhando força no mundo inteiro. Entretanto, o uso de Redes Neurais pode resultar em custos reduzidos, especialmente quando se considera que apenas um dispositivo móvel, como um smartphone, é suficiente para a captura e análise de imagens. No artigo Utilização de IA para Controle de Pragas na Agricultura \cite{santos2022}, observa-se o uso de Aprendizado de máquina. Esse projeto tem como objetivo utilizar soluções de Aprendizado de máquina para coletar e analisar imagens que mostram as condições das folhas da soja e identificar se há ou não alguma praga com base na imagem. Temos também \textit{EfficientNet} é uma arquitetura específica de Redes Neurais Convolucionais (CNN) desenvolvida para otimizar o desempenho em tarefas de visão computacional. O projeto Método de detecção de doenças da mandioca baseado no \textit{EfficientNet} utiliza esse mesmo método e se aprofunda no uso de métodos de aprendizagem profunda. O foco principal do projeto mencionado é categorizar doenças foliares usando imagens \cite{saini2023}.

De forma semelhante, o projeto Classificação de doenças foliares de mandioca, orientado por Aprendizagem Profunda \cite{gao2021}, propõe usar o espaço de cores HSV(Matiz, Saturação, Valor) além do \textit{EfficientNet} para realizar a tarefa de identificação da praga na plantação de mandioca. Converter imagens para o espaço HSV(Matiz, Saturação, Valor) como parte do pré-processamento pode simplificar o treinamento de modelos de Aprendizado de máquina e \textit{EfficientNet}, especialmente quando o modelo precisa aprender características específicas da cor. O Aprendizado de máquina mostrado anteriormente nos artigos demonstra ser uma boa opção quando se trata de avaliar imagens; porém, a Aprendizagem profunda possui uma maior capacidade de capturar padrões complexos. Sendo assim, o projeto Classificação de Bacteriose nas Folhas da Mandioca visa utilizar a Aprendizagem profunda para identificar a bactéria \textit{Xanthomonas phaseoli}. Como existem diversas bactérias semelhantes a \textit{Xanthomonas phaseoli}, a estrutura fornecida pela Aprendizado de máquina pode não ser suficiente para diferenciar essas formas da bacteriose, resultando em uma menor precisão na identificação. Portanto, a adoção de Aprendizagem profunda avançada se mostra essencial para aumentar a acurácia do diagnóstico, reduzir erros de classificação e fornecer dados confiáveis para o monitoramento das plantações. Esse enfoque permite intervenções mais rápidas e eficazes, contribuindo para a preservação da produtividade agrícola e para o manejo sustentável das culturas de mandioca. Além disso, a aplicação dessa tecnologia auxilia pesquisadores e agricultores a entender melhor a dinâmica da doença, identificando padrões de disseminação e permitindo estratégias preventivas mais eficientes, o que representa um avanço significativo para a ciência agrícola e para a segurança alimentar.

\newpage

\section*{METODOLOGIA} \label{sect:metodologia}

A metodologia adotada neste projeto foi estruturada para garantir precisão, replicabilidade e aplicabilidade prática dos resultados. O foco central é o uso de Aprendizagem Profunda (Deep Learning) na análise de imagens de folhas de mandioca, visando a detecção da bactéria Xanthomonas phaseoli pv. manihotis. O processo foi organizado em etapas sequenciais que vão desde a coleta de dados até a integração do modelo em um aplicativo funcional.

O sistema traz as seguintes proposta principais: (1) construção do conjunto de dados, (2) pré-processamento das imagens, (3) treinamento e validação do modelo de Aprendizagem Profunda, (4) testes com imagens inéditas e (5) integração ao aplicativo e ao sistema de mapeamento geográfico.
\begin{enumerate}
\item Construção do Conjunto de Dados: O conjunto de dados utilizado no projeto sera composto por imagens reais de folhas de mandioca coletadas em campos.

\item Pré-processamento das Imagens: Em seguida, serão aplicadas técnicas de aumento de dados com o objetivo de aumentar a variabilidade do conjunto de dados e melhorar a capacidade de generalização do modelo.

\item Treinamento e Validação do Modelo: O modelo foi baseado em uma Rede Neural Convolucional utilizando a arquitetura EfficientNet, devido à sua alta precisão em tarefas de visão computacional e bom desempenho em dispositivos com pouca capacidade de processamento.

\item Testes e Validação com Imagens Inéditas: Após o treinamento, o modelo sera testado utilizando imagens que não fizeram parte de nenhuma etapa anterior, garantindo avaliação real da capacidade de generalização.

\item Integração ao Aplicativo e Mapeamento Geográfico: O modelo sera treinado e integrado ao aplicativo desenvolvido, permitindo que usuários capturem ou enviem imagens das folhas de mandioca. Cada análise é armazenada juntamente com coordenadas geográficas obtidas pelo dispositivo, possibilitando a construção de um mapa de calor com as regiões mais afetadas pela bacteriose.

Esse processo permite não apenas a detecção da doença, mas também o monitoramento contínuo das plantações, contribuindo para intervenções rápidas e redução de perdas agrícolas.
\end{enumerate}

\begin{figure}[H]
    \centering
    \includegraphics[width=0.7\textwidth]{Illustrations/fluxograma.jpeg}
    \caption{Exemplo de fluxograma do sistema proposto}
    \label{fig:exemplo1}
\end{figure}
\newpage





\printbibliography

%% Elementos pós-textuais (opcionais): Apêndice e Anexo
%Caso for utilizar, basta retirar o símbolo de % na frente do comando
%\input{./Extras/post-textual}

%% Fim do documento
\end{document}