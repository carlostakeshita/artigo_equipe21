O presente estudo tem como objetivo principal o desenvolvimento de um sistema baseado em Redes Neurais para identificar e mapear a infecção por \textit{Xanthomonas phaseoli pv. manihotis} em folhas de mandioca. Com base nesse objetivo geral, foram definidos os objetivos específicos do projeto, que buscam detalhar as etapas e resultados esperados da pesquisa:

\begin{enumerate}
\item Desenvolver um sistema baseado em Redes Neurais capaz de realizar a detecção e classificação do grau de infecção das folhas de mandioca, utilizando técnicas de Aprendizagem Profunda para análise de imagens com alta precisão.
\item Integrar os dados de infecção com informações geográficas, possibilitando o mapeamento da distribuição da bactéria na região do Vale do Ribeira e fornecendo subsídios para o monitoramento territorial da praga.
\item Desenvolver um aplicativo interativo, que permita aos usuários capturar ou enviar imagens das folhas e visualizar, de forma intuitiva, o grau de infecção detectado pelo sistema.
\item Fornecer informações estratégicas para o monitoramento e controle da bactéria em tempo real, contribuindo para a redução de danos nas plantações, a manutenção da produtividade agrícola e o suporte à tomada de decisão de agricultores e técnicos da região.
\end{enumerate}

\newpage