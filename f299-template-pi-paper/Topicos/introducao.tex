A Organização das Nações Unidas (ONU) tem como proposta os Objetivos de Desenvolvimento Sustentável (ODS), que visam promover um equilíbrio entre crescimento econômico, inclusão social e proteção ambiental. Dentre esses objetivos, destaca-se o ODS 15 — “Vida Sobre a Terra”, cujo propósito é proteger, recuperar e promover o uso sustentável dos ecossistemas terrestres, além de gerir de forma sustentável as florestas, combater a desertificação, deter e reverter a degradação da terra e conter a perda da biodiversidade. Essa meta reforça a importância da preservação dos recursos naturais e a necessidade de práticas agrícolas mais sustentáveis.

O Brasil possui grande destaque como um dos principais países exportadores agrícolas do planeta, representando um papel essencial na segurança alimentar global. Em 2023, o país registrou uma alta de 4,8\% nas exportações agrícolas em relação ao ano anterior, atingindo um valor total de US\$166,55 bilhões, segundo dados do Ministério da Agricultura e Pecuária \cite{mudancas2024}. Essa expansão demonstra a relevância e o dinamismo do agronegócio brasileiro, que se apoia tanto em tecnologias inovadoras quanto em práticas tradicionais de cultivo.
Pesquisas realizadas pela Companhia Nacional de Abastecimento (CONAB) — instituição que tem como finalidade regular o estoque de alimentos, auxiliar a agricultura familiar e controlar oscilações de renda — mostram que o Brasil é o quarto maior exportador de mandioca na América Latina. O Estado de São Paulo, por sua vez, apresenta uma área plantada de 51 mil hectares, posicionando-se como o sexto maior produtor de mandioca do país.

Na região do Vale do Ribeira, localizada no sul do estado, encontra-se um número significativo de comunidades produtoras de mandioca, especialmente entre povos quilombolas, que cultivam e comercializam a produção de forma local. Essas comunidades estão distribuídas ao longo da Bacia do Rio Ribeira de Iguape, em áreas remotas e cobertas pela vegetação da Mata Atlântica, considerada uma das regiões de grande biodiversidade do planeta. Desde os primórdios da ocupação, ainda no século XVIII, os quilombolas têm mantido uma relação histórica de dependência e sustentabilidade com o cultivo de alimentos essenciais, como arroz, milho, mandioca e feijão, garantindo sua subsistência e preservando práticas agrícolas tradicionais que reforçam o equilíbrio entre o homem e o meio ambiente.
As bacterioses — doenças originadas por bactérias fitopatogênicas — ocupam posição de destaque entre as pragas agrícolas, sendo capazes de causar grandes prejuízos às lavouras quando não são rapidamente identificadas e controladas \cite{fialho1999}. Entre os diversos patógenos, destaca-se o \textit{Xanthomonas phaseoli pv. manihotis}, que atrai atenção significativa da comunidade científica em virtude de seu comportamento vascular e sistêmico, características que lhe conferem um alto potencial de causar danos severos às plantações de mandioca, especialmente em regiões tropicais. Essa bactéria tem despertado relevância científica mundial devido ao impacto direto que exerce sobre a produção alimentar e econômica \cite{mansfield2012}.

Estudos indicam que a bactéria \textit{Xanthomonas} afeta aproximadamente 30\% da produção global de mandioca, podendo reduzir em até 70\% o peso das raízes quando o controle da doença não é realizado adequadamente \cite{fialho1999}. Tais perdas representam um grave obstáculo à sustentabilidade agrícola e à segurança alimentar em diversas regiões produtoras. Um dos principais desafios enfrentados atualmente é a escassez de informações detalhadas sobre a incidência dessas bactérias, uma vez que, embora existam outros patógenos capazes de infectar a mandioca, a literatura científica ainda é limitada quanto à extensão e ao comportamento dessas infecções \cite{chaves2021}.
Com base nas pesquisas apresentadas, torna-se evidente a preocupação crescente com a ocorrência de bactérias em plantações agrícolas e seus impactos diretos na produção e na economia do setor. Com o objetivo de identificar a presença da bactéria e realizar o mapeamento de sua incidência, foi proposta a criação de um aplicativo apoiado em Redes Neurais Artificiais (RNA).

A Aprendizagem Profunda, subárea da IA, consiste em técnicas capazes de treinar redes neurais profundas para reconhecer padrões complexos e aprender continuamente a partir de grandes volumes de dados. Após o processo de treinamento, essas redes se tornam aptas a executar tarefas semelhantes às realizadas por seres humanos, como a análise e interpretação de imagens com alta precisão.
Assim, o aplicativo desenvolvido tem como finalidade monitorar o estado das plantas de mandioca e identificar, com base em probabilidade, o grau de infecção pela bactéria \textit{Xanthomonas}, fornecendo informações que auxiliem no controle e na prevenção da doença. Essa abordagem tecnológica representa um avanço significativo no monitoramento agrícola, contribuindo para minimizar perdas de produção, aumentar a eficiência no manejo de pragas e promover práticas agrícolas mais sustentáveis, em consonância com os Objetivos de Desenvolvimento Sustentável da ONU.

\newpage