O uso da IA para resolução de problemas complexos vem ganhando força no mundo inteiro. Entretanto, o uso de Redes Neurais pode resultar em custos reduzidos, especialmente quando se considera que apenas um dispositivo móvel, como um smartphone, é suficiente para a captura e análise de imagens. No artigo Utilização de IA para Controle de Pragas na Agricultura \cite{santos2022}, observa-se o uso de Aprendizado de máquina. Esse projeto tem como objetivo utilizar soluções de Aprendizado de máquina para coletar e analisar imagens que mostram as condições das folhas da soja e identificar se há ou não alguma praga com base na imagem. Temos também \textit{EfficientNet} é uma arquitetura específica de Redes Neurais Convolucionais (CNN) desenvolvida para otimizar o desempenho em tarefas de visão computacional. O projeto Método de detecção de doenças da mandioca baseado no \textit{EfficientNet} utiliza esse mesmo método e se aprofunda no uso de métodos de aprendizagem profunda. O foco principal do projeto mencionado é categorizar doenças foliares usando imagens \cite{saini2023}.

De forma semelhante, o projeto Classificação de doenças foliares de mandioca, orientado por Aprendizagem Profunda \cite{gao2021}, propõe usar o espaço de cores HSV(Matiz, Saturação, Valor) além do \textit{EfficientNet} para realizar a tarefa de identificação da praga na plantação de mandioca. Converter imagens para o espaço HSV(Matiz, Saturação, Valor) como parte do pré-processamento pode simplificar o treinamento de modelos de Aprendizado de máquina e \textit{EfficientNet}, especialmente quando o modelo precisa aprender características específicas da cor. O Aprendizado de máquina mostrado anteriormente nos artigos demonstra ser uma boa opção quando se trata de avaliar imagens; porém, a Aprendizagem profunda possui uma maior capacidade de capturar padrões complexos. Sendo assim, o projeto Classificação de Bacteriose nas Folhas da Mandioca visa utilizar a Aprendizagem profunda para identificar a bactéria \textit{Xanthomonas phaseoli}. Como existem diversas bactérias semelhantes a \textit{Xanthomonas phaseoli}, a estrutura fornecida pela Aprendizado de máquina pode não ser suficiente para diferenciar essas formas da bacteriose, resultando em uma menor precisão na identificação. Portanto, a adoção de Aprendizagem profunda avançada se mostra essencial para aumentar a acurácia do diagnóstico, reduzir erros de classificação e fornecer dados confiáveis para o monitoramento das plantações. Esse enfoque permite intervenções mais rápidas e eficazes, contribuindo para a preservação da produtividade agrícola e para o manejo sustentável das culturas de mandioca. Além disso, a aplicação dessa tecnologia auxilia pesquisadores e agricultores a entender melhor a dinâmica da doença, identificando padrões de disseminação e permitindo estratégias preventivas mais eficientes, o que representa um avanço significativo para a ciência agrícola e para a segurança alimentar.

\newpage